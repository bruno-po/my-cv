%!TEX TS-program = xelatex
%!TEX encoding = UTF-8 Unicode
% Awesome CV LaTeX Template for CV/Resume
%
% This template has been downloaded from:
% https://github.com/posquit0/Awesome-CV
%
% Author:
% Claud D. Park <posquit0.bj@gmail.com>
% http://www.posquit0.com
%
%
% Adapted to be an Rmarkdown template by Mitchell O'Hara-Wild
% 23 November 2018
%
% Template license:
% CC BY-SA 4.0 (https://creativecommons.org/licenses/by-sa/4.0/)
%
%-------------------------------------------------------------------------------
% CONFIGURATIONS
%-------------------------------------------------------------------------------
% A4 paper size by default, use 'letterpaper' for US letter
\documentclass[11pt,a4paper,]{awesome-cv}

% Configure page margins with geometry
\usepackage{geometry}
\geometry{left=1.4cm, top=.8cm, right=1.4cm, bottom=1.8cm, footskip=.5cm}


% Specify the location of the included fonts
\fontdir[fonts/]

% Color for highlights
% Awesome Colors: awesome-emerald, awesome-skyblue, awesome-red, awesome-pink, awesome-orange
%                 awesome-nephritis, awesome-concrete, awesome-darknight

\colorlet{awesome}{awesome-red}

% Colors for text
% Uncomment if you would like to specify your own color
% \definecolor{darktext}{HTML}{414141}
% \definecolor{text}{HTML}{333333}
% \definecolor{graytext}{HTML}{5D5D5D}
% \definecolor{lighttext}{HTML}{999999}

% Set false if you don't want to highlight section with awesome color
\setbool{acvSectionColorHighlight}{true}

% If you would like to change the social information separator from a pipe (|) to something else
\renewcommand{\acvHeaderSocialSep}{\quad\textbar\quad}

\def\endfirstpage{\newpage}

%-------------------------------------------------------------------------------
%	PERSONAL INFORMATION
%	Comment any of the lines below if they are not required
%-------------------------------------------------------------------------------
% Available options: circle|rectangle,edge/noedge,left/right

\name{Bruno Pinheiro de Oliveira}{}

\address{Peruíbe/SP - Brasil}

\mobile{+55 11 983070201}
\email{\href{mailto:bruno.lahunmen@gmail.com}{\nolinkurl{bruno.lahunmen@gmail.com}}}
\github{bruno-po}
\linkedin{brunopinheirooliveira}

% \gitlab{gitlab-id}
% \stackoverflow{SO-id}{SO-name}
% \skype{skype-id}
% \reddit{reddit-id}

\quote{Data Scientist}

\usepackage{booktabs}

\providecommand{\tightlist}{%
	\setlength{\itemsep}{0pt}\setlength{\parskip}{0pt}}

%------------------------------------------------------------------------------



% Pandoc CSL macros
\newlength{\cslhangindent}
\setlength{\cslhangindent}{1.5em}
\newlength{\csllabelwidth}
\setlength{\csllabelwidth}{2em}
\newenvironment{CSLReferences}[3] % #1 hanging-ident, #2 entry spacing
 {% don't indent paragraphs
  \setlength{\parindent}{0pt}
  % turn on hanging indent if param 1 is 1
  \ifodd #1 \everypar{\setlength{\hangindent}{\cslhangindent}}\ignorespaces\fi
  % set entry spacing
  \ifnum #2 > 0
  \setlength{\parskip}{#2\baselineskip}
  \fi
 }%
 {}
\usepackage{calc}
\newcommand{\CSLBlock}[1]{#1\hfill\break}
\newcommand{\CSLLeftMargin}[1]{\parbox[t]{\csllabelwidth}{\honortitlestyle{#1}}}
\newcommand{\CSLRightInline}[1]{\parbox[t]{\linewidth - \csllabelwidth}{\honordatestyle{#1}}}
\newcommand{\CSLIndent}[1]{\hspace{\cslhangindent}#1}

\begin{document}

% Print the header with above personal informations
% Give optional argument to change alignment(C: center, L: left, R: right)
\makecvheader

% Print the footer with 3 arguments(<left>, <center>, <right>)
% Leave any of these blank if they are not needed
% 2019-02-14 Chris Umphlett - add flexibility to the document name in footer, rather than have it be static Curriculum Vitae


%-------------------------------------------------------------------------------
%	CV/RESUME CONTENT
%	Each section is imported separately, open each file in turn to modify content
%------------------------------------------------------------------------------



\hypertarget{skills}{%
\section{Skills}\label{skills}}

\begin{cventries}
    \cventry{Programação}{}{}{}{\begin{cvitems}
\item Python (avançado), R (avançado), SQL (avançado)
\end{cvitems}}
    \cventry{Ferramentas}{}{}{}{\begin{cvitems}
\item Jupyter notebook, Rstudio e RMarkdown, git, Docker, AWS cloud, Airflow (básico), QGIS, Anaconda (conda envs), Linux
\end{cvitems}}
    \cventry{Frameworks}{}{}{}{\begin{cvitems}
\item pyspark, pandas, numpy, sklearn, statsmodels, pytorch, pycaret, tidyverse, tidymodels, mlr
\end{cvitems}}
    \cventry{Expertise}{}{}{}{\begin{cvitems}
\item modelos supervisionados e não supervisionados, deep learning, NLP, dashboards, story telling, análise espacial, indicadores, automatização de processos, deploy e monitoramento de modelos, data literacy
\end{cvitems}}
    \cventry{Domínio de área}{}{}{}{\begin{cvitems}
\item Políticas Públicas, Meios de pagamento, Risco de crédito, Produtos financeiros, Renda Variável
\end{cvitems}}
    \cventry{Idiomas}{}{}{}{\begin{cvitems}
\item português (nativo), espanhol (fluente), inglês (avançado)
\end{cvitems}}
\end{cventries}

\hypertarget{experiuxeancia}{%
\section{Experiência}\label{experiuxeancia}}

\begin{cventries}
    \cventry{Senior Data Scientist}{XP Inc}{São Paulo/SP}{Out/21--Jan/23}{\begin{cvitems}
\item Parte do time de desenvolvimento da AIA (Assistente de Inteligência Artificial da Clear)
\item Modelagem de churn e criação de jornada de retenção de novos clientes BMF
\end{cvitems}}
    \cventry{Data Scientist, Credit Modeling}{Stone Co}{São Paulo/SP}{Out/19--Set/21}{\begin{cvitems}
\item Dono e responsável (desenvolvimento e deploy) pelo modelo de behaviour score e pelo monitoramento da cobrança
\item Clusterização dos contratos em cobrança
\item Contribuição no desenvolvimento do modelo de credit score
\item Contribuição na modelagem de Expected Loss - EAD, PD, LGD e EL
\item Capacitações internas sobre programação e análise de dados em Python
\item Desenvolvimento do primeiro motor de crédito
\item Criação e automatização de rotinas de dados em ambiente AWS
\item Criação de cases e julgamento do I Stone Data Challenge
\end{cvitems}}
    \cventry{Instrutor de ciência de dados}{GETIP - OIPP - Vertuno - CAHS (EACH/USP)}{São Paulo/SP}{Set/18--Out/19-Out/20}{\begin{cvitems}
\item Desenvolvimento de conteúdo e aplicação de curso de análise programática de dados com o R para alunos e professores na Semana de Gestão de Políticas Públicas
da USP
\end{cvitems}}
    \cventry{Estagiário}{Controladoria Geral do Município - Prefeitura de São Paulo}{São Paulo/SP}{Mai/17--Dez/17}{\begin{cvitems}
\item Passei pelas áreas de transparência ativa e transparência passiva. Na transparência passiva trabalhei com abertura de dados, o que envolveu
a criação de rotinas de extração de dados das bases das Prefeitura, manipulação, limpeza e estruturação dos dados para divulgação no Portal
de Dados Abertos
\item Na transparência passiva trabalhei assessorando a produção de respostas aos pedidos de acesso à informação, o que também envolvia raspar
dados das bases dos órgãos da Prefeitura, além do monitoramento de indicadores
\item Nos dois setores também trabalhei com a produção de relatórios de gestão, sistematizando dados e informações para a prestação de contas
à sociedade
\end{cvitems}}
    \cventry{Pesquisador bolsista}{GETIP - Grupo de Estudos em Tecnologias e Inovações em Gestão Pública (EACH/USP)}{São Paulo/SP}{Ago/15--Dez/19}{\begin{cvitems}
\item Pesquisa sobre acessibilidade espacial no município de São Paulo, com caráter de iniciação científica, sob orientação do Prof. Dr. Alexandre
  Ribeiro Leichsenring. O desenvolvimento da pesquisa foi baseado na modelagem de dados públicos para a construção de indicadores, envolvendo um conjunto de procedimentos de programação linear e estatística: manipulação de bases de dados espaciais, rotina de captura de informações espaciais usando as API’s da Google, criação de mapas e gráficos, geração de estatísticas descritivas e cálculo de indicadores de acessibilidade espacial. Criamos uma metodologia para avaliar a acessibilidade de sistemas de atenção básica em saúde e criei um pacote no R (asha R Package) para
a replicação do método no contexto da política pública de saúde
\item Pesquisa sobre uso de tecnologias e dados no controle social da adminitração pública no Brasil, com caráter de iniciação científica, sob orientação do Prof. Dr. José Carlos Vaz. Os resultados derivaram em dois artigos publicados e apresentados no X Congresso CONSAD de Administração Pública
\end{cvitems}}
\end{cventries}

\hypertarget{educauxe7uxe3o}{%
\section{Educação}\label{educauxe7uxe3o}}

\begin{cventries}
    \cventry{Bacharelado em Gestão de Políticas Públicas}{Universidade de São Paulo / EACH}{São Paulo/SP}{Fev/2020--Dez/2023}{}\vspace{-4.0mm}
\end{cventries}

\textbf{Capacitações}

\begin{cventries}
    \cventry{AceleraDev DataScience}{Codenation}{codenation.dev}{2020}{}\vspace{-4.0mm}
    \cventry{Introdution to probability and data}{Duke University}{Coursera}{2019}{}\vspace{-4.0mm}
    \cventry{R Programming: Avanced Analytics in R for Data Science}{Super Data Science}{Udemy}{2019}{}\vspace{-4.0mm}
    \cventry{Programming for Everybody (Getting Start With Python)}{Coursera}{Michigan University}{2015}{}\vspace{-4.0mm}
    \cventry{Inteligência geográfica para governo aberto e planejamento territorial}{EAESP/FGV}{São Paulo/SP}{2019}{}\vspace{-4.0mm}
\end{cventries}

\hypertarget{publicauxe7uxf5es}{%
\section{Publicações}\label{publicauxe7uxf5es}}

\begin{cventries}
    \cventry{Pinheiro, B; Leichsenring, A}{Análise da acessibilidade espacial à atenção básica de saúde no município de São Paulo}{SIICUSP/USP}{out/18}{}\vspace{-4.0mm}
    \cventry{Pinheiro, B}{Plataformas de democracia eletrônica: um retrato para gestores públicos}{X Congresso CONSAD de Gestão Pública}{jul/17}{}\vspace{-4.0mm}
    \cventry{D`Amaral, G; Pinheiro, B}{Caracterização da participação digital no planejamento urbano do município de São Paulo}{X Congresso CONSAD de Gestão Pública}{jul/17}{}\vspace{-4.0mm}
    \cventry{Pinheiro, B; Vaz, J.C.}{Impactos do uso de Tecnologias de Informação no Controle Social da Administração Pública no Brasil}{SIICUSP/USP}{out/16}{}\vspace{-4.0mm}
\end{cventries}

\hypertarget{outras-atividades}{%
\section{Outras atividades}\label{outras-atividades}}

\begin{cventries}
    \cventry{}{Organizações}{}{}{\begin{cvitems}
\item Vice-presidente - Movimento Sócio-Ambiental Caminho das Águas, Ago/2016 a Ago/2018
\item Diretor e voluntário da Ecosurfi, out/2003 a ago/2013
\end{cvitems}}
    \cventry{}{Participação}{}{}{\begin{cvitems}
\item Representante da REJUMA no Comitê Assessor do Órgão Gestor da Política Nacional de Educação Ambiental (PNEA), Ago/2008 a Ago/2010
\end{cvitems}}
\end{cventries}


\label{LastPage}~
\end{document}
